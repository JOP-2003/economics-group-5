%%%%%%%%%%%%%%%%%%%%%%%%%%%%%%%%%%%%%%%%%%%%%%%%%%%%%%%%%%%%%%%%%%%%%%%%%%%%%%%
%
% witseiepaper-2005.tex
%
%                       Ken Nixon (12 October 2005)
%
%                       Sample Paper for ELEN417/455 2005
%
%%%%%%%%%%%%%%%%%%%%%%%%%%%%%%%%%%%%%%%%%%%%%%%%%%%%%%%%%%%%%%%%%%%%%%%%%%%%%%%%

\documentclass[10pt,onecolumn]{witseiepaper}

%
% All KJN's macros and goodies (some shameless borrowing from SPL)
\usepackage{KJN}
\usepackage{soul}

%
% PDF Info
%
\ifpdf
\pdfinfo{
/Title (INSTRUCTIONS AND STYLE GUIDELINES FOR THE PREPARATION OF FINAL YEAR LABORATORY PROJECT PAPERS : 2005 VERSION)
/Author (Ken J Nixon)
/CreationDate (D:200309251200)
/ModDate (D:200510121530)
/Subject (ELEN417/455 Paper Format, 2005)
/Keywords (ELEN417, ELEN455, paper, instructions, style guidelines, laboratory project)
}
\fi

%%%%%%%%%%%%%%%%%%%%%%%%%%%%%%%%%%%%%%%%%%%%%%%%%%%%%%%%%%%%%%%%%%%%%%%%%%%%%%%
\begin{document}


\title{MACRO-ECONOMIC MODEL OF AN IDEALISED ECONOMY}

\author{GROUP 5
\thanks{School of Electrical \& Information Engineering, University of the
Witwatersrand, Private Bag 3, 2050, Johannesburg, South Africa}
}


%%%%%%%%%%%%%%%%%%%%%%%%%%%%%%%%%%%%%%%%%%%%%%%%%%%%%%%%%%%%%%%%%%%%%%%%%%%%%%%
%
\abstract{The purpose of this document is to provide an easy-to-use
template/style sheet to enable authors to prepare papers in the correct format
and style for the final year laboratory project. This document may be
downloaded from the School of Electrical and Information Engineering web site
and can be used as a template. To ensure conformity of appearance it is
essential that these instructions are followed. The abstract should be limited
to 50-200 words, which should concisely summarise the paper.}

\keywords{Four to six key words in alphabetical order, separated by commas.}


\maketitle
\pagestyle{empty}


%%%%%%%%%%%%%%%%%%%%%%%%%%%%%%%%%%%%%%%%%%%%%%%%%%%%%%%%%%%%%%%%%%%%%%%%%%%%%%%
%
\section{INTRODUCTION}

The objective of this project is to develop a simplified yet representative model of the economy that captures the circular flow of money and goods between agents. The model is constructed to be dynamic in nature, enabling it to respond and adapt to external disturbances.  
This report begins by outlining the fundamental assumptions necessary for the functioning of an economy. It then provides a structured description of the model, detailing the mechanisms of interaction among the various agents. The subsequent sections present the results obtained from the model, followed by a critical analysis of its performance, reliability, and limitations.  
The overarching aim of this project is to advance an understanding of macroeconomic dynamics while simultaneously evaluating alternative approaches to system modelling.

\section{ASSUMPTIONS}

To construct a simplified yet meaningful representation of the economy, several assumptions are made.  
The model considers four key agents—households, firms, government, and banks—as sufficient to capture the essential flows of goods, money, fiscal policy, and monetary policy \cite{wolf2013multi}. Although real economies are more complex, this simplification allows for clarity without losing the core dynamics.  
It is further assumed that not all output will be consumed, allowing for surpluses and shortages to emerge. This reflects realistic fluctuations in demand and supply rather than imposing constant balance.  
The availability of factors of production is taken to determine the productive capacity of firms, with all resources assumed to be fully employed in production. This ensures output is limited only by resource endowments, even though inefficiencies are possible in reality.  
Finally, the economy is assumed to exhibit a natural tendency toward equilibrium, adjusting after disturbances through shifts in prices, output, or resource allocation. This provides a workable basis for analyzing responses to external shocks despite the persistence of disequilibria in real economies.  
Together, these assumptions provide a manageable framework for modelling macroeconomic interactions while maintaining analytical simplicity.

\hl{assume no illegal commerce, or corruption}


\section{Model Description}

\subsection{Households}
Households form the foundation of the economy by supplying the factors of production (labour, land, and capital) to firms in exchange for wages, rent, and dividends. These incomes enable households to demand goods and services produced by firms, thus completing the circular flow of income and expenditure. In addition, households contribute to government revenue through the payment of taxes, which are exchanged for essential public services such as education, healthcare, and security.  

Households also play a critical role in financial intermediation by depositing savings into the banking sector. These savings are not idle: banks redistribute them as loans and investments, thereby stimulating production and innovation. In return, households earn interest on their deposits, which encourages further saving. Decisions made by households—whether to consume, save, or withdraw labour—directly shape the economy’s growth trajectory and stability.

\subsection{Firms}
Firms act as the productive engines of the economy. They hire factors of production from households, paying wages, rent, and profits in return. Through the transformation of inputs into outputs, firms create goods and services that households and the government demand.  

Beyond production, firms are subject to taxation, which channels resources back to the government. In exchange, they benefit from public infrastructure such as roads, power grids, and communication systems, which reduce costs and enhance efficiency. Firms also rely on the banking sector for credit and investment to expand capacity, modernise operations, and pursue innovation. Their decisions—whether to reinvest profits, expand employment, or automate production—have lasting consequences on employment, income distribution, and the broader pace of economic development.

\subsection{Government}
The government stands as both regulator and participant in the economy. By taxing households and firms, it mobilises resources to provide essential public goods and services that cannot be efficiently supplied by markets alone. These include education, healthcare, national defence, and large-scale infrastructure projects.  

Furthermore, the government sets and implements fiscal policy, which influences aggregate demand through taxation, spending, and transfers. For example, expansionary policies can stimulate consumption and investment during recessions, while contractionary policies can cool an overheating economy. The government’s choices not only affect immediate economic stability but also shape long-term growth prospects, income equality, and social welfare.

\subsection{Banks}
Banks and other financial intermediaries form the core of the monetary system. They channel household savings into productive investments by lending to firms and financing government debt. By managing the money supply and interest rates, banks play a pivotal role in controlling inflation, stabilising prices, and fostering economic growth.  

Through monetary policy—such as altering reserve requirements, adjusting policy rates, or conducting open market operations—banks influence credit availability and liquidity. Their actions ripple across all sectors: lower interest rates may encourage households to borrow and consume, while higher rates may restrain borrowing but strengthen savings and currency stability.

\subsection{External Effects}
The model also accounts for external shocks and influences beyond domestic control. Foreign investment can inject capital, technology, and expertise into domestic firms, enhancing productivity and competitiveness in global markets. However, reliance on external capital may create vulnerabilities to global financial cycles.  

Conversely, negative shocks—such as widespread labour strikes—can disrupt production, reduce firm revenues, and suppress household incomes. These effects can feed back into reduced tax revenues and weakened demand, amplifying economic instability. Such external factors highlight the importance of resilient institutions and policies capable of absorbing shocks and sustaining long-term growth.

\section{RESULTS}

\section{ANALYSIS and DISCUSSION}

\section{CONCLUSION}


%%%%%%%%%%%%%%%%%%%%%%%%%%%%%%%%%%%%%%%%%%%%%%%%%%%%%%%%%%%%%%%%%%%%%%%%%%%%%%%
%
%\nocite{*}
\bibliographystyle{witseie}
\bibliography{sample}

%{\tiny \vfill \hfill \today \hspace{5mm} witseie-paper-2003.\TeX}

\end{document}

" vim: ts=4
" vim: tw=78
" vim: autoindent
" vim: shiftwidth=4
