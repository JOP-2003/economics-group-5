%%%%%%%%%%%%%%%%%%%%%%%%%%%%%%%%%%%%%%%%%%%%%%%%%%%%%%%%%%%%%%%%%%%%%%%%%%%%%%%
%
% witseiepaper-2005.tex
%
%                       Ken Nixon (12 October 2005)
%
%                       Sample Paper for ELEN417/455 2005
%
%%%%%%%%%%%%%%%%%%%%%%%%%%%%%%%%%%%%%%%%%%%%%%%%%%%%%%%%%%%%%%%%%%%%%%%%%%%%%%%%

\documentclass[10pt,twocolumn]{witseiepaper}

%
% All KJN's macros and goodies (some shameless borrowing from SPL)
\usepackage{KJN}
\usepackage{soul}

%
% PDF Info
%
\ifpdf
\pdfinfo{
/Title (INSTRUCTIONS AND STYLE GUIDELINES FOR THE PREPARATION OF FINAL YEAR LABORATORY PROJECT PAPERS : 2005 VERSION)
/Author (Ken J Nixon)
/CreationDate (D:200309251200)
/ModDate (D:200510121530)
/Subject (ELEN417/455 Paper Format, 2005)
/Keywords (ELEN417, ELEN455, paper, instructions, style guidelines, laboratory project)
}
\fi

%%%%%%%%%%%%%%%%%%%%%%%%%%%%%%%%%%%%%%%%%%%%%%%%%%%%%%%%%%%%%%%%%%%%%%%%%%%%%%%
\begin{document}


\title{MACRO-ECONOMIC MODEL OF AN IDEALISED ECONOMY}

\author{GROUP 5
\thanks{School of Electrical \& Information Engineering, University of the
Witwatersrand, Private Bag 3, 2050, Johannesburg, South Africa}
}


%%%%%%%%%%%%%%%%%%%%%%%%%%%%%%%%%%%%%%%%%%%%%%%%%%%%%%%%%%%%%%%%%%%%%%%%%%%%%%%
%
\abstract{The purpose of this document is to provide an easy-to-use
template/style sheet to enable authors to prepare papers in the correct format
and style for the final year laboratory project. This document may be
downloaded from the School of Electrical and Information Engineering web site
and can be used as a template. To ensure conformity of appearance it is
essential that these instructions are followed. The abstract should be limited
to 50-200 words, which should concisely summarise the paper.}

\keywords{Four to six key words in alphabetical order, separated by commas.}


\maketitle
\pagestyle{empty}


%%%%%%%%%%%%%%%%%%%%%%%%%%%%%%%%%%%%%%%%%%%%%%%%%%%%%%%%%%%%%%%%%%%%%%%%%%%%%%%
%
\section{INTRODUCTION}

The objective of this project is to develop a simplified yet representative model of the economy that captures the circular flow of money and goods between agents. The model is constructed to be dynamic in nature, enabling it to respond and adapt to external disturbances.  
This report begins by outlining the fundamental assumptions necessary for the functioning of an economy. It then provides a structured description of the model, detailing the mechanisms of interaction among the various agents. The subsequent sections present the results obtained from the model, followed by a critical analysis of its performance, reliability, and limitations.  
The overarching aim of this project is to advance an understanding of macroeconomic dynamics while simultaneously evaluating alternative approaches to system modelling.

\section{ASSUMPTIONS}

\hl{all bought is consumed, economy want to be in equalibrum, no wastage of FOP}

\section{MODEL DESCRIPTION}

\section{RESULTS}

\section{ANALYSIS and DISCUSSION}

\section{CONCLUSION}


%%%%%%%%%%%%%%%%%%%%%%%%%%%%%%%%%%%%%%%%%%%%%%%%%%%%%%%%%%%%%%%%%%%%%%%%%%%%%%%
%
%\nocite{*}
\bibliographystyle{witseie}
\bibliography{sample}

%{\tiny \vfill \hfill \today \hspace{5mm} witseie-paper-2003.\TeX}

\end{document}

" vim: ts=4
" vim: tw=78
" vim: autoindent
" vim: shiftwidth=4
